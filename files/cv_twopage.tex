\documentclass{article}

\usepackage{hyperref}
\usepackage[T1]{fontenc}
\usepackage[sc,osf]{mathpazo}
\usepackage{etaremune}

\usepackage[top=0.5in, bottom=0.5in, left=0.5in, right=0.5in]{geometry}
\usepackage{enumitem}
\usepackage{amsmath}
\begin{document}
\begin{center}
\thispagestyle{empty}
\large \textbf{Jon Green} \\
\normalsize Department of Political Science, Duke University\\
\normalsize jon.green@duke.edu $\mid$ jgreen4919.github.io $\mid$ (434) 806-3531
\end{center}

%%%%%%%%%%%%%%%%%%%%%%%%%%%%%%%%%%%%%%%%%%%%%%%%%%%%%%%%%%%%%%%
% Academic Appointments
%%%%%%%%%%%%%%%%%%%%%%%%%%%%%%%%%%%%%%%%%%%%%%%%%%%%%%%%%%%%%%%
\noindent \textbf{\underline{Academic Appointments}}\\

\noindent \textbf{Duke University}\\
\indent Assistant Professor, Department of Political Science (2023-)

\noindent \textbf{Harvard Kennedy School}\\
\indent Fellow, Shorenstein Center on Media, Politics \& Public Policy (2021-2023)

\noindent \textbf{Northeastern University}\\
\indent Senior Research Scientist, Network Science Institute (2021-2023)\\
\indent Affiliated Researcher, NULab for Text, Maps, and Networks (2022-2023) \\
\indent Part-Time Lecturer, Department of Political Science (2022-2023) \\
\indent Postdoctoral Fellow, Network Science Institute (2020-2021) \\

\noindent \textbf{\underline{Education}}\\

\noindent Ph.D. in Political Science, Ohio State University, 2020 \\
\noindent B.A. with High Honors in Political Science, Kenyon College, 2014 \\

\noindent \textbf{\underline{Current Research Interests}} \\

\noindent My current research concerns information problems in democracy, specifically in the United States. Democratic citizens need to use information to contribute to collective decisions. The tasks of curating and organizing that information is prohibitively costly for individual citizens to undertake on their own, and must necessarily be done collectively. How are these tasks collectively undertaken in public discourse, and what are the behavioral consequences? \\

\noindent One strand of my research in this area concerns the dynamics of contemporary public discourse. This includes papers examining how political coalitions develop and maintain ideologies, building on prior models of ``long" coalition formation (\textit{POQ -- Rhetorical 'What Goes With What'}, R\&R at \textit{Political Behavior}); how prior work concerning the effects of decision rules on gender (in)equality in political discussions translates to highly visible settings where discussions are norm-bound rather than rule-bound (\textit{JOP}); how individuals curate information from a variety of sources to advance their political goals (\textit{APSR, Nature, Nature  Human Behaviour}); and elite polarization and representation more broadly (\textit{PRQ, Science Advances, JQD}). A second strand of research concerns behavioral consequences of these dynamics -- how democratic citizens respond to public discourse? This manifests in a variety of settings, such as science communication (\textit{BJPS, POQ -- COVID-19 Spillover}), racial attitudes (\textit{JREP, PSQ}), and voting behavior (\textit{PNAS, POQ -- Strategic Discrimination}). \\

\noindent I have recently begun a new project that examines the consequences of recent structural transformations in our information environment, which have led citizens to increasingly encounter non-institutional sources of political information and opinion leadership. Empirical work in this area examines political and politics-adjacent podcasts, collecting large scale show- and episode-level data and developing network-based methods for surveying podcast audiences beyond the minority of respondents who listen to the biggest shows. Current work in progress from this project include understanding the high-level structure of the podcast ecosystem based on the network of which guests appear on which shows, quantifying the relative extent of intentional versus incidental exposure to political discussion, and testing prevailing theories regarding audience composition and behavior that have thus far only been empirically studied using individual (and potentially anomalous) shows. \\

\noindent \textbf{\underline{Select Recent Publications}}
	\begin{itemize}
		\item Goel, Pranav, \textbf{Jon Green}, David Lazer, and Philip Resnik. 2025. ``Using co-sharing to identify use of mainstream news for promoting potentially misleading narratives." \textit{Nature Human Behaviour}.
		\item Smith, Alyssa, \textbf{Jon Green}, Brooke Focault Welles, and David Lazer. 2025. ``Emergent structures of attention on social media are driven by amplification and triad transitivity." \textit{PNAS Nexus}, pgaf106.
		\item  Hobbs, William, and \textbf{Jon Green}. 2025. ``Categorizing topics versus inferring attitudes: a theory and method for analyzing open-ended survey responses." \textit{Political Analysis}.
		\item \textbf{Green, Jon}, Stefan McCabe, Sarah Shugars, Hanyu Chwe, Luke Horgan, Shuyang Cao, and David Lazer. 2025. ``Curation Bubbles." \textit{American Political Science Review}.
		\item \textbf{Green, Jon}. 2025. ``The Rhetorical `What Goes with What': Political Pundits and the Discursive Superstructure of Ideology in U.S. Politics." \textit{Public Opinion Quarterly}, nfae060.
		\item Naftel, Daniel, \textbf{Jon Green}, Kelsey Shoub, Jared Edgerton, Mallory Wagner, and Skyler Cranmer. 2025. ``Meet the Press: Gendered Conversational Norms in Televised Political Discussions." \textit{Journal of Politics} 87(3).
		\item \textbf{Green, Jon}, Kelsey Shoub, Rachel Blum, and Lindsey Cormack. 2024. ``Cross-Platform Partisan Positioning in Congressional Speech." \textit{Political Research Quarterly} 77(3): 653–668.
		\item Lunz Trujillo, Kristin, \textbf{Jon Green}, Alauna Safarpour, David Lazer, Jennifer Lin, and Matthew Motta. 2024. "Covid-19 Spillover Effects onto General Vaccine Attitudes." \textit{Public Opinion Quarterly} 88(1): 97-122.
		\item Chewning, Taylor K., \textbf{Jon Green}, Hans J.G. Hassell, and Matthew R. Miles. 2024. "Campaign Principal-Agent Problems: Volunteers as Faithful and Representative Agents." \textit{Political Behavior} 46: 405-426.
		\item Robertson, Ronald E., \textbf{Jon Green}, Damian Ruck, Katherine Ognyanova, Christo Wilson, and David Lazer. 2023. ``Users choose to engage with more partisan news than they are exposed to on Google Search."\textit{Nature} 618: 342-348.
		\item \textbf{Green, Jon}, Meredith Conroy, and Ciera Hammond. 2023. ``Something to Run For: Stated Motives as Indicators of Candidate Emergence." \textit{Political Behavior} 46: 1281-1301.
		\item \textbf{Green, Jon}, Brian Schaffner, and Sam Luks. 2023. "Strategic Discrimination in the 2020 Democratic Primary." \textit{Public Opinion Quarterly} 86(4): 886-898.
		\item McCabe, Stefan, \textbf{Jon Green}, Pranav Goel, and David Lazer. 2023. ``Inequalities in Online Representation: Who follows their own member of Congress on Twitter?" \textit{Journal of Quantitative Description: Digital Media}, 3.
		\begin{itemize}
		\item Winner of 2024 Best Article award, Information Technology \& Politics Section, American Political Science Association.
		\end{itemize}
		\item \textbf{Green, Jon}, and Mark H. White II. 2023. \textit{Machine Learning for Experiments in the Social Sciences.} Cambridge University Press, Elements Series in Experimental Political Science.
		\item \textbf{Green, Jon}, William Hobbs, Stefan McCabe, and David Lazer. 2022. ``Online Engagement with 2020 Election Misinformation and Turnout in the 2021 Georgia Runoff Election"  \textit{Proceedings of the National Academy of Sciences} 119(34): e2115900119.
		\item  \textbf{Green, Jon}, James N. Druckman, Matthew A. Baum, David Lazer, Katherine Ognyanova, Matthew Simonson, Jennifer Lin, Mauricio Santillana, and Roy Perlis. 2022. "Using General Messages to Persuade on a Politicized Scientific Issue." \textit{British Journal of Political Science} 53(2): 698-706.
		\item \textbf{Green, Jon}. 2021. "Does Race Stop at the Water's Edge? Elites, the Public, and Support for Foreign Intervention among White U.S. Citizens over Time." \textit{Political Science Quarterly} 136(2): 339-361. 
		\item \textbf{Green, Jon}, Jared Edgerton, Daniel Naftel, Kelsey Shoub, and Skyler Cranmer. 2020. ``Elusive Consensus: Polarization in Elite Communication on the COVID-19 Pandemic." \textit{Science Advances} 6(28): eabc2717.
\end{itemize}

\noindent \textbf{\underline{Manuscripts Under Review or Publicly Available}}

\begin{itemize}

\item Wan, Allison, and \textbf{Jon Green}. "Political Pundits and the Maintenance of Ideological Coalitions." Revise and resubmit, \textit{Political Behavior}.

\item \textbf{Green, Jon}, William Minozzi, and Michael Neblo. "How We Do Things with Surveys: Recognizing Questions and Answers as 'Speech Acts' Will Improve the Science of Public Opinion." Under review.

\item \textbf{Green, Jon}. "Asymmetric Concentration in the Marketplace of Ideas." Working paper: \url{https://osf.io/gbkn5}

\item \textbf{Green, Jon}. "Informal (De-)Regulation in the Marketplace of Ideas." Working paper: \url{https://osf.io/7v4zj}

\end{itemize}


\end{document}
